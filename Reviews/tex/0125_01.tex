
\documentclass[10pt]{article}

\title{CS423 Summary: The Nucleus of a Multiprogramming System}
\author{Hongpeng Guo}
\date{\today}
\begin{document}
\maketitle

\noindent
{\bf Area:}

This paper focus on the area of multi-programming system, which can suits diverse requirements of program scheduling and resource allocation.
\\
\noindent
{\bf Problem:}

At the time of this paper. All the existing operating systems support only one processing model. Changing processing model for different programs is really expensive, if not impossible. The goal of this paper is to design a general operating system structure which support multiple processing models.
\\
\noindent
{\bf Solution:}

The key idea of the solution is the structure consisting parallel, cooperating processes. A system nucleus is designed in such system to handle the simulation of processes; communication among processes, creation, control and removal of processes.
\\
\noindent
{\bf Methodology:}

\begin{itemize}
\item
The system nucleus is seen as a software extension of hardware and manages all the processes of the system.
\item
Processes communicate with each other in the system environment to facilitate the synchronization of information during data transferring.
\item
The process hierarchy is a tree structure. The overall operating system works as the tree root. A new child process is presented as a child node to its parent process. Parent process distribute resources to its children.
\end{itemize}

\noindent
{\bf Results:}

This paper did not implement any experiments. But RC4000 operating system is designed under the paper's idea. Several measurements on RC4000 is provided in this paper, which shows merits at that time.
\\
\noindent
{\bf Takeaway:}

The system nucleus idea is the key point, which contributes to the kernel idea of modern systems. When dealing with complicated systems, it is a good idea to create a leader/ nucleus one to to the management of the whole system.

\end{document}
