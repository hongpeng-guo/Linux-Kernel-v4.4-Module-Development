
\documentclass[10pt]{article}

\title{CS423 Summary: Reflections on an Operating System Design}
\author{Hongpeng Guo}
\date{\today}
\begin{document}
\maketitle

\noindent
{\bf Area:}

This paper provides a reflection of the whole design and implementation process for the Cal system. It lies in the area of operating system design.
\\

\noindent
{\bf Problem:}

This paper did not have a specific problem, but summarize the whole design principles of the Cal system. The author illustrated several successful points which can be adopted by later system developers and also listed bad points for later system to avoid. 
\\

\noindent
{\bf Methodology:}

The system designers organized this system into protected layers. The author first introduced two important layers (1) the kernel layer and (2) the user layers. And then listed the successful and fail points for the whole system.
\\

\noindent
{\bf Solution:}

This paper did not have proper solutions for all the problems encountered during the system design process. But there are several successful trials as below:
\begin{itemize}
\item
The capability list idea performs well on access control of the system.
\item
The PPU handles system I/O really well.
\item
The extensibility of the system is good for the protected layer design principle.
\end{itemize}

\noindent
{\bf Results:}

Finally the project was terminated for lack of funds. the system was neither
efficient enough nor usable enough to be put into service by the computer center.
\\

\noindent
{\bf Takeaway:}

Successful points include (1) the use of capabilities, (2) the idea of protected layering, (3) the conversion of input-output devices into processes with a minimum of interpretation, ... etc.
Lessons learned include (1) the attempt to provide the illusion of a mapped address space on unsuitable hardware (2) the way in which the disk was incorporated into the memory hierarchy, (3) include a layer of memory in future system design process.

\end{document}
